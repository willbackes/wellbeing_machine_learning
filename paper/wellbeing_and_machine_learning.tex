\documentclass[11pt, a4paper, leqno]{article}
\usepackage{a4wide}
\usepackage[T1]{fontenc}
\usepackage[utf8]{inputenc}
\usepackage{float, afterpage, rotating, graphicx}
\usepackage{epstopdf}
\usepackage{longtable, booktabs, tabularx}
\usepackage{fancyvrb, moreverb, relsize}
\usepackage{eurosym, calc}
% \usepackage{chngcntr}
\usepackage{amsmath, amssymb, amsfonts, amsthm, bm}
\usepackage{caption}
\usepackage{mdwlist}
\usepackage{xfrac}
\usepackage{setspace}
\usepackage[dvipsnames]{xcolor}
\usepackage{subcaption}
\usepackage{minibox}
\usepackage{csvsimple}
\usepackage{booktabs}
\usepackage{pgfplotstable}

% \usepackage{pdf14} % Enable for Manuscriptcentral -- can't handle pdf 1.5
% \usepackage{endfloat} % Enable to move tables / figures to the end. Useful for some
% submissions.

\usepackage[
    natbib=true,
    bibencoding=inputenc,
    bibstyle=authoryear-ibid,
    citestyle=authoryear-comp,
    maxcitenames=3,
    maxbibnames=10,
    useprefix=false,
    sortcites=true,
    backend=biber
]{biblatex}
\AtBeginDocument{\toggletrue{blx@useprefix}}
\AtBeginBibliography{\togglefalse{blx@useprefix}}
\setlength{\bibitemsep}{1.5ex}
\addbibresource{refs.bib}

\usepackage[unicode=true]{hyperref}
\hypersetup{
    colorlinks=true,
    linkcolor=black,
    anchorcolor=black,
    citecolor=NavyBlue,
    filecolor=black,
    menucolor=black,
    runcolor=black,
    urlcolor=NavyBlue
}


\widowpenalty=10000
\clubpenalty=10000

\setlength{\parskip}{1ex}
\setlength{\parindent}{0ex}
\setstretch{1.5}


\begin{document}

\title{Comparative Analysis of Predictive Models on Human Wellbeing\thanks{William Backes, University of Bonn. Email: \href{mailto:s19wback@uni-bonn.de}{\nolinkurl{s19wback [at] uni-bonn [dot] de}}.}}

\author{William Backes}

\date{
    {\bf Preliminary -- please do not quote}
    \\[1ex]
    \today
}

\maketitle


\begin{abstract}
    Some abstract here.
\end{abstract}

\clearpage


\section{Introduction} % (fold)
\label{sec:introduction}

This project is centered around exploring the application of machine learning techniques
to predict human wellbeing, comparing traditional econometric methods, such as Ordinary
Least Squares (OLS), with modern algorithms like Least Absolute Shrinkage and Selection
Operator (LASSO), Random Forests (RF), and Gradient Boosting (GB). Inspired by the work
of Oparina et al. (2023), the aim is to assess the effectiveness of various models in
predicting human wellbeing. The project was created using the template from \citet{GaudeckerEconProjectTemplates}
% section introduction (end)

\section{Methods} % (fold)
\label{sec:methods}

\subsection{Data}
The dataset for this project is sourced from the German Socio-Economic Panel (SOEP),
covering the years 2010 to 2018. This timeframe aligns with the original paper, and
flexibility is maintained to consider other years as long as data remains available.
The SOEP is a representative longitudinal survey of private households in Germany. One main feature of the SOEP data is that it follows the same private households, individuals, and families every year, surveying a range of topics from living conditions, education to values and personality since 1984.

\subsection{Algorithms}

\subsection{Explanatory variables}


\input{../bld/final/descriptive_stats_continuous.tex}


\begin{figure}[htbp]
    \centering
    \includegraphics[width=0.85\textwidth]{../bld/final/histogram_life_satisfaction}
    \caption{\emph{Python:} Model predictions of the smoking probability over the lifetime. Each colored line represents a case where marital status is fixed to one of the values present in the data set.}
    \label{fig:histogram_life_satisfaction}
\end{figure}


\section{Results} % (fold)
\label{sec:results}

\subsection{Model performance}

\begin{figure}[htbp]
    \centering
    \includegraphics[width=0.85\textwidth]{../bld/final/r_squared_algorithms}
    \caption{\emph{Python:} Model predictions of the smoking probability over the lifetime. Each colored line represents a case where marital status is fixed to one of the values present in the data set.}
    \label{fig:r_squared_algorithms}
\end{figure}


\subsection{Variable importance}

\input{../bld/final/permutation_importance_table.tex}




\subsection{Wellbeing by age and income}

\begin{figure}[htbp]
    \centering
    \includegraphics[width=0.85\textwidth]{../bld/final/average_wellbeing_by_age}
    \caption{\emph{Python:} Model predictions of the smoking probability over the lifetime. Each colored line represents a case where marital status is fixed to one of the values present in the data set.}
    \label{fig:average_wellbeing_by_age}
\end{figure}

\begin{figure}[htbp]
    \centering
    \includegraphics[width=0.85\textwidth]{../bld/final/average_wellbeing_by_income}
    \caption{\emph{Python:} Model predictions of the smoking probability over the lifetime. Each colored line represents a case where marital status is fixed to one of the values present in the data set.}
    \label{fig:average_wellbeing_by_income}
\end{figure}

\section{Discussion} % (fold)
\label{sec:discussion}





\setstretch{1}
\printbibliography
\setstretch{1.5}


% \appendix

% The chngctr package is needed for the following lines.
% \counterwithin{table}{section}
% \counterwithin{figure}{section}

\end{document}
